\allowdisplaybreaks
%%% Question 4 - - - - - - - - - - - - - - - - - - - - - - - - - - - - - -
\question This question is about equational reasoning.
\begin{parts}
	
	\part[4] A well known property of the \haskellIn{map} function in Haskell is map fusion which states that, for all functions \haskellIn{f} and \haskellIn{g} and for all lists \haskellIn{xs}, the following holds:
	
	\begin{center}
		\haskellIn{map f (map g xs) == map (f . g) xs}
	\end{center}

	Using a suitable example, explain why this property would not be true if \haskellIn{f} or \haskellIn{g} were not pure functions. \droppoints
	 
	\begin{solution}
	\emph{Comprehension.} The order in which \haskellIn{f} and \haskellIn{g} are called is different for the two expressions. In the case of \haskellIn{map f (map g xs)}, \haskellIn{g} is called for all elements in \haskellIn{xs} and then \haskellIn{f} is called for all elements in the resulting list. For, \haskellIn{map (f . g) xs}, \haskellIn{f} is called immediately after \haskellIn{g} for every element of \haskellIn{xs}. If, for example, \haskellIn{f} uses some global state (like a counter) to determine its result and modifies that counter, then the results will be different.
	\end{solution}

	\part[4] Suppose that natural numbers are defined as follows along with a function for addition of natural numbers:
	\begin{small}
		\begin{verbatim}
		data Nat = Zero | Succ Nat
		
		add :: Nat -> Nat -> Nat 
		add Zero     m = m 
		add (Succ n) m = Succ (add n m)
		\end{verbatim}
	\end{small}

	%\begin{subparts}
		Addition should be an associative binary operation:   
		\begin{center}
			\haskellIn{add (add x y) z == add x (add y z)}
		\end{center}
		Prove that this property holds. \droppoints
	
		\begin{solution}
			\emph{Application/Bookwork. 1 mark for base case. 3 marks for inductive step.} This property was proved in the lectures. The proof is as follows, by induction on $x$: 
			\begin{displaymath}
			\begin{array}{cl}
			\expr{\mathit{add}~\mathit{Zero}~(\mathit{add~y~z})}
			\hint{applying $\mathit{add}$}
			\expr{\mathit{add~y~z}}
			\hint{unapplying $\mathit{add}$}
			\lastexpr{\mathit{add}~(\mathit{add}~\mathit{Zero}~y)~z}
			\end{array}
			\end{displaymath}
			The inductive step is for $\mathit{Succ}~n$:
			\begin{displaymath}
			\begin{array}{cl}
			\expr{\mathit{add}~(\mathit{Succ}~x)~(\mathit{add~y~z})}
			\hint{applying $\mathit{add}$}
			\expr{\mathit{Succ}~(\mathit{add}~x~(\mathit{add}~y~z))}
			\hint{induction hypothesis}
			\expr{\mathit{Succ}~(\mathit{add}~(\mathit{add}~x~y)~z)}
			\hint{unapplying $\mathit{add}$}
			\expr{(\mathit{add}~(\mathit{Succ}~\mathit{add}~x~y)~z)}
			\hint{unapplying $\mathit{add}$}
			\lastexpr{(\mathit{add}~(\mathit{add}~(\mathit{Succ}~x)~y)~z)}
			\end{array}
			\end{displaymath}
		\end{solution}

	\part[5] Consider the following property about \haskellIn{replicate} and \haskellIn{(++)}:
	\begin{center}
		\haskellIn{replicate n x ++ replicate m x == replicate (add n m) x}
	\end{center}
	Assume that \haskellIn{replicate} is defined as follows:
	\begin{small}
		\begin{verbatim}
		replicate :: Nat -> a -> [a]
		replicate Zero     x = []
		replicate (Succ n) x = x : replicate n x
		\end{verbatim}
	\end{small}
	Prove that the property shown above holds. \droppoints

	\begin{solution}
		\emph{Application.} The proof is by induction on $n$. The base case is for $\mathit{Zero}$:
		\begin{displaymath}
		\begin{array}{cl}
		\expr{\mathit{replicate}~\mathit{Zero}~x \append \mathit{replicate}~m~x}
		\hint{applying $\mathit{replicate}$}
		\expr{\hslist{} \append \mathit{replicate}~m~x}
		\hint{applying $\append$}
		\expr{\mathit{replicate}~m~x}
		\hint{unapplying $\mathit{add}$}
		\lastexpr{\mathit{replicate}~(\mathit{add}~\mathit{Zero}~m)~x}
		\end{array}
		\end{displaymath}
		The inductive case is for $\mathit{Succ}~n$:
		\begin{displaymath}
		\begin{array}{cl}
		\expr{\mathit{replicate}~(\mathit{Succ}~n)~x \append \mathit{replicate}~m~x}
		\hint{applying $\mathit{replicate}$}
		\expr{(x : \mathit{replicate}~n~x) \append \mathit{replicate}~m~x}
		\hint{applying $\append$}
		\expr{x : (\mathit{replicate}~n~x \append \mathit{replicate}~m~x)}
		\hint{induction hypothesis}
		\expr{x : (\mathit{replicate}~(\mathit{add}~n~m)~x)}
		\hint{unapplying $\mathit{replicate}$}
		\expr{\mathit{replicate}~(\mathit{Succ}~(\mathit{add}~n~m))~x}
		\hint{unapplying $\mathit{add}$}
		\lastexpr{\mathit{replicate}~(\mathit{add}~(\mathit{Succ}~n)~m)~x}
		\end{array}
		\end{displaymath}
	\end{solution}

	\part[12] Prove the following property which states the length of concatenating a list of lists is the same as calculating the length of each individual list and then calculating the sum of the results:   
	\begin{center}
		\haskellIn{length (concat xss) == sum (map length xss)}
	\end{center}
	You may assume that the usual properties of integers and arithmetic operations on them have already been proved. \droppoints

	\begin{solution}
		\emph{Application.} This question has two proofs hidden in one and there are 6 marks available for each proof (2 for base case + 4 for inductive step each). 
		
		Let us begin with the proof for the property shown by induction on $\mathit{xss}$. The base case is for $\hslist{}$:
		
		\begin{displaymath}
		\begin{array}{cl}
		\expr{\mathit{length}~(\mathit{concat}~\hslist{})}
		\hint{applying $\mathit{concat}$}
		\expr{\mathit{length}~\hslist{}}
		\hint{applying $\mathit{length}$}
		\expr{0}
		\hint{unapplying $\mathit{sum}$}
		\expr{\mathit{sum}~\hslist{}}
		\hint{unapplying $\mathit{map}~\mathit{length}$}
		\lastexpr{\mathit{sum}~(\mathit{map}~\mathit{length}~\hslist{})}
		\end{array}
		\end{displaymath}
		
		Then we can prove the inductive case for $\mathit{xs}:\mathit{xss}$:
		
		\begin{displaymath}
		\begin{array}{cl}
		\expr{\mathit{length}~(\mathit{concat}~(\mathit{xs}:\mathit{xss}))}
		\hint{applying $\mathit{concat}$}
		\expr{\mathit{length}~(\mathit{xs} \append \mathit{concat}~\mathit{xss})}
		\hint{lemma proved below}
		\expr{\mathit{length}~\mathit{xs} + \mathit{length}~(\mathit{concat}~\mathit{xss})}
		\hint{induction hypothesis}
		\expr{\mathit{length}~\mathit{xs} + \mathit{sum}~(\mathit{map}~\mathit{length}~\mathit{xss})}
		\hint{unapplying $\mathit{sum}$}
		\expr{\mathit{sum}~(\mathit{length}~\mathit{xs} : \mathit{map}~\mathit{length}~\mathit{xss})}
		\hint{unapplying $\mathit{map}~\mathit{length}$}
		\lastexpr{\mathit{sum}~(\mathit{map}~\mathit{length}~(\mathit{xs}:\mathit{xss}))}
		\end{array}
		\end{displaymath}
		
		As shown in the proof above, a lemma about how $\mathit{length}$ distributes over $\append$ is required:
		
		\begin{center}
			\begin{small}
				\begin{verbatim}
				length (xs ++ ys) == length xs + length ys
				\end{verbatim}
			\end{small}
		\end{center}
		
		The proof is by induction on $\mathit{xs}$. First the base case for $\hslist{}$:
		\begin{displaymath}
		\begin{array}{cl}
		\expr{\mathit{length}~(\hslist{} \append \mathit{ys})}
		\hint{applying $\append$}
		\expr{\mathit{length}~\mathit{ys}}
		\hint{identity of addition}
		\expr{0 + \mathit{length}~\mathit{ys}}
		\hint{unapplying $\mathit{length}$}
		\lastexpr{\mathit{length}~\hslist{} + \mathit{length}~\mathit{ys}}
		\end{array}
		\end{displaymath}
		
		Then the inductive case for $x:\mathit{xs}$:
		
		\begin{displaymath}
		\begin{array}{cl}
		\expr{\mathit{length}~((x:\mathit{xs}) \append \mathit{ys})}
		\hint{applying $\append$}
		\expr{\mathit{length}~(x : (\mathit{xs} \append \mathit{ys}))}
		\hint{applying $\mathit{length}$}
		\expr{1 + \mathit{length}~(\mathit{xs} \append \mathit{ys})}
		\hint{induction hypothesis}
		\expr{1 + \mathit{length}~\mathit{xs} + \mathit{length}~\mathit{ys}}
		\hint{unapplying $\mathit{length}$}
		\lastexpr{\mathit{length}~(x:\mathit{xs}) + \mathit{length}~\mathit{ys}}
		\end{array}
		\end{displaymath}
		This concludes the proof.
	\end{solution}
	
\end{parts}
